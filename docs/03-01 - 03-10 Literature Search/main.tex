\documentclass{article}

\usepackage{natbib}
\usepackage{parskip}
\usepackage{url}

\title{Literature Values Search}
\author{Thomas Locks}

\begin{document}
\noindent
\maketitle

\section{Escape Velocity from Dust Disk}
\citet{holland2014treatise}, in Section 6.1.2.2, gives an escape velocity in relation to a gravitational focusing factor. See Equation \ref{eq:grav_focusing}.

\begin{equation}
    F_g = 1 + \frac{v_{esc}^2}{v_{rel}^2}
    \label{eq:grav_focusing}
\end{equation}

In Equation \ref{eq:grav_focusing}, $v_{esc}$ is the body's escape velocity, and $v_{rel}$ is the escape velocity of other particles in the body's environment.

It is unclear whether this is the right approach. Can a dust disk be considered a `body' in this context?

\section{Lifetime of Biological Material on Rocks in Space}
% Assume that small rocks are much more common than larger rocks, and the largest rocks are no more massive than the Moon.
\citet{ginsburg2018galactic} gives


\begin{equation}
    f_{survived} = e^{-\frac{t}{\tau_{bio}}}
    \label{eq:survival_frac_to_lifetime}
\end{equation}

In Equation \ref{eq:survival_frac_to_lifetime}, $\tau_{bio}$ is the biological lifetime, and $t$ is the time the biological material spends inside the rock.

\citet{adams2022transfer} looks promising. It gives a distribution of rocky objects by mass:

\begin{equation}
    \frac{dN}{dm} = Am^{-p}
    \label{eq:rock_mass_dist}
\end{equation}

In Equation \ref{eq:rock_mass_dist}, $m$ is the mass, $N$ is the number of rocks, and $p$ has values between 1 and 2. When the size-mass distribution is `determined by collisional processes', $p \approx 1.8$. This can be linked to rock radius if we have an assumed density for these rocks. $A$ is defined in \citet{adams2022transfer}.

\url{https://astrostatistics.psu.edu/datasets/asteroid_dens.html#:~:text=However%2C%20the%20internal%20structure%20of,structures%20will%20have%20lower%20density.} gives a density of asteroids ('Oumuamua is said to have the same density -- see the Wikipedia page) of 3--5 gcm$^{-3}$.

\citet{adams2022transfer} gives a minimum mass for the rocks to harbour biological life as 10 kg. 

\citet{adams2022transfer} also gives a minimum rock size for the rocks to harbour biological life as a few centimetres. 

\citet{adams2022transfer} also gives a value for the fraction of remaining bodies as a function of time since capture:

\begin{equation}
    f_{survived}(t) = \frac{1}{1 + (t/\tau)^{8/5}}
    \label{eq:survival_frac_adams}
\end{equation}

In Equation \ref{eq:survival_frac_adams}, $\tau = 0.84$ Myr; $t$ is the time since the capture of the object. This value is derived from a fit to experimental data. 

\subsection{Lifetime of These Rocks}

% Rocks from M-dwarfs will have been around for longer (or conversely, rocks from G-stars will have been around for less time), because of the lifetime difference between these stars.


\bibliographystyle{agsm}
\bibliography{bibliography}

\end{document}